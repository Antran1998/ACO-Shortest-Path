\section{Motivation}

This work addresses the path planning problem on grid maps using Ant Colony Optimization (ACO). Classical ACO implementations for grid-based navigation can suffer from slow convergence, path jaggedness (unsuitable for continuous motion), and suboptimal exploitation/exploration balance. These limitations affect real-world use-cases such as mobile robot navigation, autonomous ground vehicles in structured environments, and automated guided vehicles (AGVs) in warehouses.

Prior approaches include standard ACO variants, heuristic-only planners (A*, Dijkstra), and sampling-based planners (RRT, PRM). Strengths and weaknesses:
- A* / Dijkstra: exact on discrete grids but can produce unnatural paths and do not generalize well when using learned heuristics.
- Sampling planners: good for continuous spaces but may be inefficient in dense discrete grids.
- Baseline ACO: good at exploring multiple candidate paths but sensitive to parameter choices and produces discrete, jagged routes.

The team selected a set of complementary improvements (cone-shaped pheromone initialization, adaptive pheromone/heuristic factors, division of labor among ants, backtracking heuristics, B-spline smoothing, and coordinated parameter fine-tuning) because together they:
1. Introduce gentle directional guidance without overconstraining exploration.
2. Adapt search behavior during optimization to avoid premature convergence.
3. Allow role specialization to improve search efficiency.
4. Produce geometrically shorter, smooth, and executable trajectories suitable for real robotic platforms.