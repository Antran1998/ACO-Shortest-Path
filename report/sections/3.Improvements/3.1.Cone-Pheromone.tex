\section{Cone Pheromone (Ân)}

\subsection{Idea}
Initialize pheromone distribution with a cone-shaped bias pointed toward the goal. This provides gentle directional guidance without forcing a fixed path.

\subsection{Implementation}
Compute initial pheromone for each node/edge using a combination of a normalized cone term and an inverse-distance term:
\[
\tau_0 = \tau_{\text{base}} + c \cdot \frac{|x-y|}{L} + \frac{1}{d}
\]
where $c$ is a small coefficient, $L$ is map scale and $d$ is Euclidean distance to goal.

\subsection{Benefits}
Faster convergence, lower variance across runs, maintains exploration while guiding ants toward promising regions.

\subsection{Experimental results}
Insert per-map quantitative results here: min/mean/std path length and runtime compared to baseline.