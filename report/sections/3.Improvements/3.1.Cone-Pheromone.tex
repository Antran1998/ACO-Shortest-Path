\section{Cone Pheromone}
\sloppy

\subsection{Idea}
Initialize pheromone distribution with a cone-shaped bias pointed toward the goal. This provides gentle directional guidance without forcing a fixed path.

\subsection{Implementation}
Compute initial pheromone for each node/edge using a combination of a normalized cone term and an inverse-distance term. In the implementation (see \texttt{aco\_enhancement/ant\_colony\_enhancement.py}) the initialization is:
\begin{equation*}
\tau_0 = \tau_{\mathrm{base}} + c\cdot\frac{|x-y|}{L} + \frac{1}{d+\epsilon}
\end{equation*}
where:
\begin{itemize}
    \item \texttt{tau\_base} is the configured \texttt{initial\_pheromone} (base pheromone level),
    \item \texttt{c} is the cone coefficient (implemented with a default value \texttt{c=0.09}),
    \item \texttt{|x-y|} is the coordinate difference used as a simple directional cue (the code uses the absolute difference between the node's column and row indices),
    \item \texttt{L} is the map scale (the code uses the map dimension \texttt{map\_len = in\_map.shape[0]}),
    \item \texttt{d} is the Euclidean distance from the edge's target node to the goal, and
    \item \texttt{epsilon} is a small constant to avoid division by zero (\texttt{EPSILON = 1e-6} in the code).
\end{itemize}

Concretely, the code computes the cone term as \texttt{(0.09 * coordinate\_diff) / map\_len} and the inverse-distance term as \texttt{1.0 / (d + EPSILON)}. The resulting value is added to \texttt{initial\_pheromone} and written to the edge as \texttt{edge['Pheromone']}; edge probabilities are initialized to \texttt{0.0}.

\subsection{Benefits}
Faster convergence, lower variance across runs, maintains exploration while guiding ants toward promising regions.

\subsection{Experimental results}
We compare the standard ACO (Basic ACO) to the cone-biased initialization (Cone Pheromone (O1)) on four maps using average path length (Mean Len, with standard deviation) and computation time. The results are summarized in Table~\ref{tab:cone-results}.

% Visualization for Map a
\begin{figure}[H]
    \centering
    \begin{subfigure}[b]{0.45\textwidth}
        \centering
        \includegraphics[width=\linewidth]{images/O1/visu-map1-basic_aco.png}
        \caption{Map a - Basic ACO}
    \end{subfigure}
    \hfill
    \begin{subfigure}[b]{0.45\textwidth}
        \centering
        \includegraphics[width=\linewidth]{images/O1/visu-map1-m2-cone_pheromone_o1.png}
        \caption{Map a - Cone Pheromone (O1)}
    \end{subfigure}
    \caption{Visualization of paths on Map a.}
    \label{fig:cone-path-visualization-map1}
\end{figure}

% Visualization for Map b
\begin{figure}[H]
    \centering
    \begin{subfigure}[b]{0.45\textwidth}
        \centering
        \includegraphics[width=\linewidth]{images/O1/visu-map3-basic_aco.png}
        \caption{Map b - Basic ACO}
    \end{subfigure}
    \hfill
    \begin{subfigure}[b]{0.45\textwidth}
        \centering
        \includegraphics[width=\linewidth]{images/O1/visu-map3-m2-cone_pheromone_o1.png}
        \caption{Map b - Cone Pheromone (O1)}
    \end{subfigure}
    \caption{Visualization of paths on Map b.}
    \label{fig:cone-path-visualization-map3}
\end{figure}

% Visualization for Map c
\begin{figure}[H]
    \centering
    \begin{subfigure}[b]{0.45\textwidth}
        \centering
        \includegraphics[width=\linewidth]{images/O1/visu-map7-basic_aco.png}
        \caption{Map c - Basic ACO}
    \end{subfigure}
    \hfill
    \begin{subfigure}[b]{0.45\textwidth}
        \centering
        \includegraphics[width=\linewidth]{images/O1/visu-map7-m2-cone_pheromone_o1.png}
        \caption{Map c - Cone Pheromone (O1)}
    \end{subfigure}
    \caption{Visualization of paths on Map c.}
    \label{fig:cone-path-visualization-map8}
\end{figure}

% Visualization for Map d
\begin{figure}[H]
    \centering
    \begin{subfigure}[b]{0.45\textwidth}
        \centering
        \includegraphics[width=\linewidth]{images/O1/visu-map8-basic_aco.png}
        \caption{Map d - Basic ACO}
    \end{subfigure}
    \hfill
    \begin{subfigure}[b]{0.45\textwidth}
        \centering
        \includegraphics[width=\linewidth]{images/O1/visu-map8-m2-cone_pheromone_o1.png}
        \caption{Map d - Cone Pheromone (O1)}
    \end{subfigure}
    \caption{Visualization of paths on Map d.}
    \label{fig:cone-path-visualization-map7}
\end{figure}

\begin{table}[H]
    \centering
    \caption{Comparison of Basic ACO and Cone Pheromone (O1) on different maps}
    \label{tab:cone-results}
    \begin{tabular}{|c|c|c|c|}
        \hline
    	\textbf{Map Name} & \textbf{Algorithm} & \textbf{Mean path length} & \textbf{Computational time (s)} \\
        \hline
        \multirow{2}{*}{Map a} & Basic ACO & 6.04 $\pm$ 0.54 & 0.166 \\
         & Cone Pheromone (O1) & 5.24 $\pm$ 0.00 & 0.213 \\
        \hline
        \multirow{2}{*}{Map b} & Basic ACO & 89.15 $\pm$ 7.56 & 5.766 \\
         & Cone Pheromone (O1) & 49.58 $\pm$ 1.59 & 4.075 \\
        \hline
        \multirow{2}{*}{Map c} & Basic ACO & 37.30 $\pm$ 2.06 & 2.158 \\
         & Cone Pheromone (O1) & 23.26 $\pm$ 0.34 & 1.469 \\
        \hline
        \multirow{2}{*}{Map d} & Basic ACO & 85.11 $\pm$ 10.29 & 5.603 \\
         & Cone Pheromone (O1) & 48.02 $\pm$ 2.17 & 2.645 \\
        \hline
    \end{tabular}
\end{table}

The experimental results indicate that the cone-initialized pheromone substantially improves solution quality and often reduces runtime, especially in cluttered, crowded, and maze-like environments.

\subsubsection*{Map a}
Both methods reach the optimal corridor; O1 is consistently optimal (5.24 $\pm$ 0.00) versus Basic ACO (6.04 $\pm$ 0.54). Runtime is comparable (0.213 s vs. 0.166 s).

\subsubsection*{Map b}
O1 finds much shorter paths (49.58 $\pm$ 1.59) than Basic ACO (89.15 $\pm$ 7.56) and is faster on average (4.075 s vs. 5.766 s), showing strong guidance without over-bias.

\subsubsection*{Map c}
O1 significantly outperforms the baseline (48.02 $\pm$ 2.17 vs. 85.11 $\pm$ 10.29) with lower time (2.645 s vs. 5.603 s), demonstrating robustness in complex layouts.

\subsubsection*{Map d}
O1 is markedly better (23.26 $\pm$ 0.34) than Basic ACO (37.30 $\pm$ 2.06) and also faster (1.469 s vs. 2.158 s), indicating faster convergence with lower variance.

\subsubsection*{Summary and Insights}
The cone-shaped initialization improves both mean path length and stability across all maps, with notable gains in challenging environments. Guidance from the cone field accelerates early exploration toward promising regions while preserving diversity, leading to shorter paths and competitive or better runtimes.