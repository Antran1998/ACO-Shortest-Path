\section{Backtracking}

\subsection{Idea}
Standard ACO implementations suffer from the deadlock problem: when an ant encounters a node where all neighboring cells are either obstacles or already visited, the ant becomes trapped and must abandon its current search attempt. This premature termination wastes computational resources and reduces the algorithm's exploration efficiency, especially in maps with narrow corridors, dead-ends or complex obstacle configurations.

To address this limitation, an intelligent backtracking mechanism is introduced that allows ants to perform controlled, multi-step backtracking when encountering dead-ends. Instead of immediately restarting from scratch, ants can revisit recent decision points nodes where multiple path choices were available and explore previously unexplored alternatives. This approach transforms deadlocks from terminal failures into temporary setbacks, significantly improving path discovery rates in constrained environments.

\subsection{Implementation}
The backtracking mechanism is implemented through three complementary data structures and algorithms integrated into the \texttt{Ant} class:

\textbf{Decision Stack:} Each ant maintains a stack of decision points recorded during path traversal. A decision point is saved whenever the ant encounters a node with multiple unvisited neighbors (line 46-53 in the code). Each stack entry stores:
\begin{itemize}
    \item The node position where the decision was made
    \item The path length at that point (for restoration)
    \item The set of unexplored neighbor nodes
\end{itemize}

\textbf{Multi-Step Backtracking Algorithm:} When an ant becomes trapped (all neighbors visited or blocked), the \texttt{backtrack\_to\_decision\_point} method (lines 55-77) is invoked. The algorithm iterates through the decision stack in reverse chronological order, searching for a decision point that still has unexplored alternatives. When such a point is found:
\begin{enumerate}
    \item The ant's path is truncated to the decision point
    \item Nodes visited after that point are removed from the visited set
    \item The ant resumes exploration from the decision point with fresh options
\end{enumerate}

If no valid backtrack point exists (all decision points exhausted), the ant has encountered a true structural deadlock and terminates.

\textbf{Global Tabu List:} To prevent repeated failures at problematic nodes, a colony-level tabu list tracks nodes that have caused persistent deadlocks. When an ant exhausts all backtracking attempts at a node, that node is added to the global tabu list and avoided by all subsequent ants in the current iteration (lines 179, 204).

\textbf{Adaptive Path Quality Penalty:} During next-node selection, the algorithm applies an adaptive penalty to discourage paths that repeatedly backtrack or show poor progress toward the goal. The penalty factor ranges from 0.05 to 0.2 based on the current path quality metric, which combines distance to goal with path straightness (lines 250-257).

\textbf{Backtracking Limits:} Each ant is restricted to a maximum of 10 backtracking operations per search attempt (\texttt{max\_backtracks} parameter) to prevent infinite loops and ensure computational efficiency.

\subsection{Benefits}
The backtracking mechanism provides several key advantages:

\textbf{Increased Success Rate:} Ants can recover from temporary dead-ends and discover valid paths in maps where the baseline algorithm would fail. This is particularly valuable in environments with narrow passages or maze-like structures.

\textbf{Reduced Wasted Computation:} Instead of discarding partially-explored paths, ants reuse accumulated knowledge about the search space. Backtracking preserves the pheromone information along successful path segments while only discarding failed branches.

\textbf{Intelligent Exploration:} The decision stack naturally prioritizes recent choices, implementing a depth-first search bias that complements ACO's probabilistic exploration. This helps ants thoroughly explore promising corridors before abandoning them.

\textbf{Adaptive Learning:} The global tabu list enables colony-level learning, where information about structural deadlocks is shared across all ants, preventing redundant failures.

\subsection{Experimental results}
The backtracking mechanism was evaluated on four benchmark maps. Each configuration was tested with 30 independent runs using standard parameters: 50 ants, 100 iterations, evaporation rate of 0.15, and initial pheromone concentration of $1 \times 10^{-4}$. Table~\ref{tab:backtracking_results} presents the comparative performance of backtracking against the baseline ACO.

\begin{table}[H]
\centering
\caption{Comparison of Base ACO and Backtracking mechanism across different maps}
\label{tab:backtracking_results}
\begin{tabular}{|c|c|c|c|}
\hline
\textbf{Map Name} & \textbf{Algorithm} & \textbf{Mean path length} & \textbf{Computational time (s)} \\
\hline
\multirow{2}{*}{Map a} & Base ACO & $5.24 \pm 0.00$ & 0.110 \\
 & Backtracking & $5.24 \pm 0.00$ & 0.127 \\
\hline
\multirow{2}{*}{Map b} & Base ACO & $47.77 \pm 1.08$ & 1.980 \\
 & Backtracking & $52.59 \pm 3.73$ & 2.546 \\
\hline
\multirow{2}{*}{Map c} & Base ACO & $44.59 \pm 0.50$ & 1.388 \\
 & Backtracking & $47.23 \pm 2.93$ & 1.541 \\
\hline
\multirow{2}{*}{Map d} & Base ACO & $22.98 \pm 0.08$ & 0.653 \\
 & Backtracking & $23.08 \pm 0.26$ & 0.792 \\
\hline
\end{tabular}
\end{table}

\textbf{Path Quality Analysis:} Backtracking preserves optimal minimum path length across all maps but shows degraded mean path length in complex environments.

\textbf{Variance and Stability:} Increased standard deviation indicates reduced consistency on complex maps due to stochastic backtracking decisions.

\textbf{Computational Overhead:} The mechanism introduces 15--30\% runtime overhead; overhead scales with environment complexity.

\textbf{Performance Trade-offs and Applicability:} Standalone backtracking is most valuable when success rate is critical or combined with complementary improvements.