\section{Combination of Improvements (Proposed Method)}
\sloppy

\subsection{Idea}
Combine all proposed improvements (O1--O6) into a single pipeline to leverage complementary strengths: cone pheromone initialization, adaptive heuristic scheduling, division of labor, deadlock/backtracking, path smoothing, and the restructured control flow.

\subsection{Implementation}
All modules are enabled jointly in the enhanced ACO implementation. The decision policy uses cone-initialized pheromone and adaptive $(\alpha'(n), \beta'(n))$, ants are assigned roles (explorer/exploiter), deadlocks trigger backtracking and re-placement, and the algorithm follows the restructured loop with termination-time smoothing.

\subsection{Benefits}
The combined method aims to:
\begin{itemize}
    \item Improve path quality and reliability across varied environments.
    \item Maintain competitive or reduced runtime via efficient control flow and recovery.
    \item Reduce variance by coordinating exploration and exploitation throughout iterations.
\end{itemize}

\subsection{Experimental Results}
We compare Basic ACO, individual improvements (O1--O6), and the full combination (Proposed Method (Full)). Metrics are mean path length (with standard deviation) and computation time; best values per map are highlighted in bold. Results are summarized in Table~\ref{tab:combination-results}.

\begin{table}[H]
    \centering
    \small
    \caption{Final comparison across all methods on four maps}
    \label{tab:combination-results}
    \begin{tabular}{|c|c|c|c|}
        \hline
        \textbf{Map Name} & \textbf{Algorithm} & \textbf{Mean path length} & \textbf{Computational time (s)} \\
        \hline
        \multirow{8}{*}{Map a} 
            & Basic ACO & 6.04 $\pm$ 0.64 & 0.209 \\
            & Cone Pheromone (O1) & 5.24 $\pm$ 0.00 & 0.180 \\
            & Adaptive Heuristic (O2) & \textbf{5.24 $\pm$ 0.00} & \textbf{0.170} \\
            & Division of Labor (O3) & 5.24 $\pm$ 0.00 & 0.235 \\
            & Backtracking (O4) & 5.24 $\pm$ 0.00 & 0.265 \\
            & Path Smoothing (O5) & 5.24 $\pm$ 0.00 & 0.190 \\
            & New Algorithm Flow (O6) & 5.24 $\pm$ 0.00 & 0.183 \\
            & \textbf{Proposed Method (Full)} & \textbf{5.24 $\pm$ 0.00} & 0.198 \\
        \hline
        \multirow{8}{*}{Map b} 
            & Basic ACO & 88.48 $\pm$ 6.95 & 3.056 \\
            & Cone Pheromone (O1) & \textbf{49.21 $\pm$ 1.68} & \textbf{2.025} \\
            & Adaptive Heuristic (O2) & 86.27 $\pm$ 23.06 & 3.263 \\
            & Division of Labor (O3) & 59.30 $\pm$ 3.47 & 2.359 \\
            & Backtracking (O4) & 55.29 $\pm$ 5.26 & 2.556 \\
            & Path Smoothing (O5) & 82.45 $\pm$ 22.17 & 3.315 \\
            & New Algorithm Flow (O6) & 89.73 $\pm$ 25.78 & 3.939 \\
            & \textbf{Proposed Method (Full)} & 49.68 $\pm$ 2.15 & 2.629 \\
        \hline
        \multirow{8}{*}{Map c} 
            & Basic ACO & 85.08 $\pm$ 9.43 & 3.119 \\
            & Cone Pheromone (O1) & 47.98 $\pm$ 2.31 & \textbf{1.442} \\
            & Adaptive Heuristic (O2) & 61.19 $\pm$ 9.80 & 1.740 \\
            & Division of Labor (O3) & 50.81 $\pm$ 2.40 & 1.462 \\
            & Backtracking (O4) & \textbf{47.92 $\pm$ 3.52} & 1.583 \\
            & Path Smoothing (O5) & 61.28 $\pm$ 8.91 & 1.674 \\
            & New Algorithm Flow (O6) & 62.75 $\pm$ 11.02 & 1.740 \\
            & \textbf{Proposed Method (Full)} & 46.26 $\pm$ 2.05 & 1.503 \\
        \hline
                \multirow{8}{*}{Map d} 
            & Basic ACO & 37.11 $\pm$ 2.37 & 1.300 \\
            & Cone Pheromone (O1) & \textbf{23.20 $\pm$ 0.32} & 0.764 \\
            & Adaptive Heuristic (O2) & 23.72 $\pm$ 1.01 & 0.777 \\
            & Division of Labor (O3) & 23.30 $\pm$ 0.36 & \textbf{0.758} \\
            & Backtracking (O4) & 23.60 $\pm$ 0.76 & 0.848 \\
            & Path Smoothing (O5) & 24.01 $\pm$ 1.22 & 0.812 \\
            & New Algorithm Flow (O6) & 23.96 $\pm$ 1.48 & 0.802 \\
            & \textbf{Proposed Method (Full)} & 23.95 $\pm$ 0.98 & 0.926 \\
        \hline
    \end{tabular}
\end{table}

The combined method delivers strong and stable performance overall, often near the best path quality while keeping runtime competitive.

\subsubsection*{Map a}
All improvements reach the optimal path (5.24 $\pm$ 0.00). The full combination is close to the fastest variants.

\subsubsection*{Map b}
Cone initialization (O1) attains the best mean length (49.21). The combined method remains near-optimal (52.19) with moderate runtime.

\subsubsection*{Map c}
The combined method is effectively tied with the best single components (O1/O4) on mean length and remains near the fastest runtime.

\subsubsection*{Map d}
O1/O3 slightly edge the best mean lengths, while the combined method is close (23.95) with acceptable time.

\subsubsection*{Summary and Insights}
Across all maps, the \emph{Proposed Method (Full)} consistently improves over Basic ACO and matches or approaches the best single-component results.