\section{Adaptive Pheromone / Heuristic Factors}

\subsection{Idea}
In the standard Ant Colony Optimization (ACO) algorithm, the pheromone influence factor $\alpha$ and heuristic influence factor $\beta$ are typically fixed throughout the optimization process. However, static parameter settings are not well suited for complex path planning problems, where different search stages demand different behaviors.

The core idea of this improvement is to dynamically adapt $\alpha$ and $\beta$ along the iteration process in order to achieve a better balance between exploration and exploitation. Specifically, lower influence of pheromone and heuristic information is preferred in early iterations to encourage exploration of diverse paths, while stronger influence is gradually introduced in later iterations to enhance exploitation and accelerate convergence toward high-quality solutions.

\subsection{Implementation}
To realize the above idea, a smooth adaptive scheduling strategy is applied to both $\alpha$ and $\beta$. Let $n$ denote the current iteration index and $N$ be the maximum number of iterations. The adaptive parameters are defined as:
\[
\alpha'(n) = \alpha + \xi \cdot \frac{1}{2}\left(\frac{n}{N}\right)^2, \quad
\beta'(n) = \beta + \xi \cdot \frac{1}{2}\left(\frac{n}{N}\right)^2
\]

where $\alpha$ and $\beta$ are the initial values, and $\xi$ is a tunable coefficient controlling the strength of adaptation.

The quadratic schedule ensures a gradual and smooth increase of parameter values. At early iterations ($n \ll N$), the increment is small, maintaining high randomness in ant decision-making. As the iteration progresses, the influence of pheromone trails and heuristic information becomes increasingly dominant, reinforcing promising paths and guiding ants toward convergence.

The adaptive parameters $\alpha'(n)$ and $\beta'(n)$ are updated at each iteration and directly applied in the state transition probability without modifying the fundamental structure of the ACO algorithm.

\subsection{Benefits}
The proposed adaptive pheromone and heuristic factor strategy provides several advantages:
\begin{itemize}
    \item It effectively reduces premature convergence by preventing early over-exploitation of suboptimal paths.
    \item It enhances global exploration capability during the initial search phase, improving the diversity of candidate solutions.
    \item It accelerates convergence in later iterations by strengthening positive feedback on high-quality paths.
    \item It introduces minimal computational overhead while significantly improving robustness and stability.
\end{itemize}

Overall, this adaptive mechanism enables the ACO algorithm to dynamically adjust its search behavior according to the optimization stage, which is particularly beneficial in complex and cluttered environments.

\subsection{Experimental Results}
The effectiveness of the adaptive pheromone and heuristic factor regulation is validated through comparative experiments with the standard fixed-parameter ACO. Under identical environmental settings and parameter initialization, performance metrics such as average path length, convergence speed, and solution variance are evaluated.

Experimental results show that the adaptive ACO consistently produces shorter paths with lower variance compared to the baseline method. Moreover, the convergence curves demonstrate that the adaptive strategy achieves faster and more stable convergence, especially in environments with dense obstacles. These results confirm that dynamic parameter adaptation significantly enhances both solution quality and robustness of the path planning process.
