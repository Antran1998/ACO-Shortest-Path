\section{Division of Labor}

\subsection{Idea}
To overcome the limitations of standard Ant Colony Optimization (ACO) where all agents follow uniform transition rules, a dynamic division of labor strategy is implemented. By mimicking natural colonies, ants are categorized into two distinct roles: soldiers and kings. Soldier ants prioritize global exploration through an $\epsilon$-greedy strategy, selecting random valid neighbors with a probability of $\epsilon = 0.2$ to prevent premature convergence to suboptimal paths. Conversely, king ants focus on path consolidation by adhering strictly to high-quality pheromone trails.

\subsection{Implementation}
The distribution of roles evolves according to a transition probability $\Lambda$, defined by the relationship between a time factor $S$ and a path variation coefficient $\theta$. This mechanism ensures a shift from a soldier-dominated exploration phase to a king-dominated exploitation phase. In the Python implementation, the \texttt{Ant} class includes a role attribute that modifies the \texttt{select\_next\_node} method to trigger either random selection or weighted probability based on the assigned role.

\subsection{Benefits}
This strategy optimizes search efficiency by dynamically balancing exploration and exploitation. It prevents the algorithm from stagnating in local optima during early stages while accelerating convergence during the final iterations.

\subsection{Experimental results}
We compare the Division of Labor strategy (O3) against the Basic ACO on four maps. Results are aggregated over repeated runs, and the table format matches the Adaptive Heuristic (O2) section for consistency.

\begin{table}[H]
    \centering
    \caption{Comparison of Basic ACO and Division of Labor (O3) on different maps}
    \label{tab:dol_performance}
    \begin{tabular}{|c|c|c|c|}
        \hline
        \textbf{Map Name} & \textbf{Algorithm} & \textbf{Mean path length} & \textbf{Computational time (s)} \\
        \hline
        \multirow{2}{*}{Map a} & Basic ACO & 6.03 $\pm$ 0.59 & 0.283 \\
         & Division of Labor (O3) & 5.24 $\pm$ 0.00 & 0.243 \\
        \hline
        \multirow{2}{*}{Map b} & Basic ACO & 89.33 $\pm$ 7.60 & 3.326 \\
         & Division of Labor (O3) & 58.99 $\pm$ 3.85 & 2.458 \\
        \hline
        \multirow{2}{*}{Map c} & Basic ACO & 86.36 $\pm$ 10.42 & 2.996 \\
         & Division of Labor (O3) & 50.57 $\pm$ 2.01 & 1.461 \\
        \hline
        \multirow{2}{*}{Map d} & Basic ACO & 37.13 $\pm$ 2.56 & 1.119 \\
         & Division of Labor (O3) & 23.18 $\pm$ 0.32 & 0.735 \\
        \hline
    \end{tabular}
\end{table}

Overall, O3 consistently shortens mean path length and reduces computation time across all maps.

% Visualization for Map a
\begin{figure}[H]
    \centering
    \begin{subfigure}[b]{0.45\textwidth}
        \centering
        \includegraphics[width=\linewidth]{images/O3/visu-map1-basic_aco.png}
        \caption{Map a - Basic ACO}
    \end{subfigure}
    \hfill
    \begin{subfigure}[b]{0.45\textwidth}
        \centering
        \includegraphics[width=\linewidth]{images/O3/visu-map1-m4-division_of_labor_o3.png}
        \caption{Map a - Division of Labor (O3)}
    \end{subfigure}
    \caption{Visualization of paths on Map a.}
    \label{fig:dol-path-visualization-map1}
\end{figure}

% Visualization for Map b
\begin{figure}[H]
    \centering
    \begin{subfigure}[b]{0.45\textwidth}
        \centering
        \includegraphics[width=\linewidth]{images/O3/visu-map3-basic_aco.png}
        \caption{Map b - Basic ACO}
    \end{subfigure}
    \hfill
    \begin{subfigure}[b]{0.45\textwidth}
        \centering
        \includegraphics[width=\linewidth]{images/O3/visu-map3-m4-division_of_labor_o3.png}
        \caption{Map b - Division of Labor (O3)}
    \end{subfigure}
    \caption{Visualization of paths on Map b.}
    \label{fig:dol-path-visualization-map3}
\end{figure}

% Visualization for Map c
\begin{figure}[H]
    \centering
    \begin{subfigure}[b]{0.45\textwidth}
        \centering
        \includegraphics[width=\linewidth]{images/O3/visu-map7-basic_aco.png}
        \caption{Map c - Basic ACO}
    \end{subfigure}
    \hfill
    \begin{subfigure}[b]{0.45\textwidth}
        \centering
        \includegraphics[width=\linewidth]{images/O3/visu-map7-m4-division_of_labor_o3.png}
        \caption{Map c - Division of Labor (O3)}
    \end{subfigure}
    \caption{Visualization of paths on Map c.}
    \label{fig:dol-path-visualization-map8}
\end{figure}

% Visualization for Map d
\begin{figure}[H]
    \centering
    \begin{subfigure}[b]{0.45\textwidth}
        \centering
        \includegraphics[width=\linewidth]{images/O3/visu-map8-basic_aco.png}
        \caption{Map d - Basic ACO}
    \end{subfigure}
    \hfill
    \begin{subfigure}[b]{0.45\textwidth}
        \centering
        \includegraphics[width=\linewidth]{images/O3/visu-map8-m4-division_of_labor_o3.png}
        \caption{Map d - Division of Labor (O3)}
    \end{subfigure}
    \caption{Visualization of paths on Map d.}
    \label{fig:dol-path-visualization-map7}
\end{figure}
