\section{Division of Labor (Thương)}

\subsection{Idea}
To overcome the limitations of standard Ant Colony Optimization (ACO) where all agents follow uniform transition rules, a dynamic division of labor strategy is implemented. By mimicking natural colonies, ants are categorized into two distinct roles: soldiers and kings. Soldier ants prioritize global exploration through an $\epsilon$-greedy strategy, selecting random valid neighbors with a probability of $\epsilon = 0.2$ to prevent premature convergence to suboptimal paths. Conversely, king ants focus on path consolidation by adhering strictly to high-quality pheromone trails.

\subsection{Implementation}
The distribution of roles evolves according to a transition probability $\Lambda$, defined by the relationship between a time factor $S$ and a path variation coefficient $\theta$. This mechanism ensures a shift from a soldier-dominated exploration phase to a king-dominated exploitation phase. In the Python implementation, the \texttt{Ant} class includes a role attribute that modifies the \texttt{select\_next\_node} method to trigger either random selection or weighted probability based on the assigned role.

\subsection{Benefits}
This strategy optimizes search efficiency by dynamically balancing exploration and exploitation. It prevents the algorithm from stagnating in local optima during early stages while accelerating convergence during the final iterations.

\subsection{Experimental results}
The performance of the Division of Labor strategy (O3) was evaluated on four grid maps ranging from simple to complex structures. Table~\ref{tab:dol_performance} details the experimental outcomes in terms of path length and computation time.

\begin{table}[H]
	\centering
	\caption{Performance of Division of Labor strategy across different environments}
	\label{tab:dol_performance}
	\begin{tabular}{l|c|c|c}
		\toprule
		\textbf{Environment} & \textbf{Min Length (m)} & \textbf{Mean Length (m)} & \textbf{Time (s)} \\
		\midrule
		Map 1 (Simple) & 5.24 & 5.24 $\pm$ 0.00 & 0.337 \\
		Map 3 (Complex) & 51.00 & 51.00 $\pm$ 1.66 & 4.859 \\
		Map 7 (Medium) & 47.46 & 47.46 $\pm$ 1.37 & 3.086 \\
		Map 8 (Medium) & 22.97 & 22.97 $\pm$ 0.00 & 1.939 \\
		\bottomrule
	\end{tabular}
\end{table}

The results demonstrate the algorithm's capability to find feasible paths across varying levels of complexity. Fig.~\ref{fig:dol_maps} visualizes the best paths generated by the Division of Labor strategy in each environment.

\begin{figure}[H]
	\centering
	\begin{subfigure}[b]{0.48\textwidth}
		\includegraphics[width=\textwidth]{images/visu-map3-m4-division_of_labor_o3-map3.png}
		\caption{Map 3}
	\end{subfigure}
	\hfill
	\begin{subfigure}[b]{0.48\textwidth}
		\includegraphics[width=\textwidth]{images/visu-map7-m4-division_of_labor_o3-map7.png}
		\caption{Map 7}
	\end{subfigure}
	
	\vspace{0.5cm}
	
	\begin{subfigure}[b]{0.48\textwidth}
		\includegraphics[width=\textwidth]{images/visu-map8-m4-division_of_labor_o3.png}
		\caption{Map 8}
	\end{subfigure}
	\hfill
	\begin{subfigure}[b]{0.48\textwidth}
		\includegraphics[width=\textwidth]{images/visu-map1-m4-division_of_labor_o3-map1.png}
		\caption{Map 1}
	\end{subfigure}
	\caption{Discrete paths generated by Division of Labor strategy across different environments.}
	\label{fig:dol_maps}
\end{figure}
