\section{Conclusion}

This work improves Ant Colony Optimization (ACO) for grid-based path planning by introducing a set of complementary algorithmic and post-processing enhancements. The project focused on: (1) cone-shaped pheromone initialization to provide gentle directional guidance; (2) adaptive pheromone/heuristic scheduling to balance exploration and exploitation over iterations; (3) division of labor to specialise ant roles for speed and robustness; (4) controlled backtracking to recover from dead-ends; (5) B-spline smoothing to produce continuous executable trajectories; and (6) coordinated parameter fine-tuning so all features work together.

Key findings
\begin{itemize}
  \item All proposed configurations maintain the optimal minimum path found on benchmark maps while improving secondary metrics (consistency, mean path quality, runtime) compared to the baseline.
  \item The cone-pheromone initialization yields the best mean path length and the lowest variance across runs, making it the best single improvement for reproducible quality.
  \item Division of labor provides the largest runtime improvement at the cost of higher variance, making it appropriate for time-critical scenarios.
  \item The combined configuration (``Mix All'') achieves the best balance between quality, consistency and runtime, demonstrating synergy among the improvements.
  \item B-spline smoothing converts discrete grid paths into smooth, high-density trajectories suitable for real-world motion controllers.
\end{itemize}

Limitations
\begin{itemize}
  \item Experiments were performed on static, known occupancy-grid maps; dynamic or partially-observed environments were not evaluated.
  \item Parameter sensitivity remains: although coordinated tuning reduces conflicts, further automated tuning (e.g. Bayesian optimization) may improve robustness.
  \item Low-level control and execution (actuator models, vehicle dynamics) are out of scope and require integration and validation on target platforms.
\end{itemize}

Practical recommendations
\begin{itemize}
  \item For quality-critical deployments (robotics, AGVs): enable cone-pheromone and Euclidean-distance fitness, apply B-spline smoothing.
  \item For time-critical applications: enable division of labor; accept slightly higher variance for faster results.
  \item For general use: enable the combined configuration (Mix All) with the tuned parameter set provided in the implementation notes and appendices.
\end{itemize}

Future work
\begin{itemize}
  \item Extend evaluation to dynamic maps and moving obstacles with online re-planning.
  \item Integrate learning-based heuristics or learned initial pheromone fields to further reduce tuning effort.
  \item Validate the full pipeline on a physical robot or high-fidelity simulator to measure real-world execution performance.
  \item Add automated parameter search and multi-objective optimization (trade-off between runtime, smoothness and path length).
\end{itemize}

In summary, the presented enhancements make ACO a more practical and robust option for grid-based path planning. The modular design allows selective activation of improvements depending on application constraints (quality vs. speed), and the smoothing stage produces trajectories ready for downstream control.